\chapter{Introduction}\label{ch:introduction}
\epigraph{
- Homer, is it the way you pictured PhD life?\\
- Yeah, pretty much, except we drove in a van solving mysteries.
}{Could be Homer Simpson}

\begin{addmargin}[2em]{2em}
This thesis links the research fields of data mining, constraint learning
and logic programming. We introduce them in turn and provide an overview of the
contributions of the thesis and
its general structure.
\end{addmargin}

The ability to write down a model and to simply push a button to get the
answer is an appealing idea. It is one of the holy grails of
Artificial Intelligence, called Declarative Programming. Today in the age of data, we would like
to write models, feed them with data and find the insight slipping
away from a human eye. What we do not really want is to write
all of tedious code. Once we have written a model, we would like
to [VERB NEEDED] a feature to it and get an updated answer. Unsurprisingly, it is
not an easy task. To tackle it we need first to introduce Data Mining
problems and then attack them with the most powerful weapon -- Logic.
The main reason behind this is simple: in this thesis, we are
concerned with Relational Data and Relational Data Mining Problems.
And Logic is the best modelling tool to handle relational data
problems. [я не знаю. последние 3 предложения - это очень strong statements, и далеко не все будут с ними согласны. я бы переформулировала гораздо менее категорично, напирая на то, что это одна из возможных опций, у которой к тому же есть плюсы 1), 2), 3). explainable AI, то-сё.]

\section{Data Mining}
Data Mining is the scientific area concerned with turning data into knowledge. 
The amount of data available and generated today is truly enormous. 
Data Mining is devoted to structuring, filtering, classifying and extracting useful patterns from this data.

The \textit{knowledge} extracted from the data can manifest itself in
multiple forms. For example, an interesting piece of data,
\textit{typically a substructure}, can be
[referred to as]/considered a \textit{pattern} \parencite{han_book}. 
The most known [NOUN NEEDED! pattern learning task?] is the
\textit{frequent pattern mining} , or pattern mining for short \parencite{survey_han}. 
At a high level it can be formalized as follows: 
let $D$ be a set of object and $\phi$ be a function of
interest, then the pattern mining problem is the problem of finding
all objects $d$ in $D$ such that $\phi(d)$ holds.
Simply speaking, we enumerate objects that have a certain property of
interest.
Various measures of relevance or ``interestingness'' of a pattern exist, which lead to an extensive research of multiple types of patterns \parencite{tias_topk}.


Many Data Mining problems, and especially pattern mining
problems, are \textbf{combinatorial problems}, which in turn have two main components to them, namely problem modelling and search: 

\begin{center}
  Combinatorial problem = Model + Search
\end{center}

Therefore, it seems natural to model pattern mining problems 
using systems designed to perform search in declarative manner, such as \acrlong{cp} or \acrlong{asp}. %(which we discuss later in detail).




\section{Constraint Learning}
Constraint Learning is the research field concerned with finding
constraints in the data \parencite{constraint_learning,QUACQ,Conacq}.
%Let us outline in simple terms, what constraint learning is about.
Similar to the pattern matching problem, we can define 
the constraint learning problem as follows:
%Explaining what constraint learning in term of the previous section:
let $D$ be a dataset and $\Phi$ be a set of interesting properties, then
then the problem is to find all properties $\phi$ in $\Phi$ such that $\phi(D)$ holds.

It is not hard to see a certain duality between the two problems, essentially:
\begin{center}
  \textbf{Patterns are constraints and constraints are patterns}.
\end{center}

In case of pattern mining, we enumerate objects on which the property
of interest holds, and in constraint learning we find interesting
properties that hold in the data (or its parts).
Therefore, we consider constraint learning and pattern mining to be interconnected, and both problems play a central role in this thesis.

\section{Logic Programming}
In his seminal work Sir Bob \textcite{kowalski} proclaimed:
\begin{center}
  Algorithm = Logic + Control.
\end{center}

%What he meant is quite simple, 
The intuition behind this statement is that
you can turn logical rules into a computational process. 
Why is this important for us? Logic deals with
relations and relational structures, it is naturally designed to
work in the relational setting. Therefore, if we can combine logic
with computation, we obtain a computational engine that can work in
our relational learning setting.

Consider an example in which there are three people: John, Peter and Elena. 
We know that people who are not allergic to fish like sushi; 
furthermore, we know that Peter is allergic to fish. 
Who likes sushi then? We can write this puzzle as a
simple logical implication:

\begin{equation*}
  \textit{likes\_sushi}(X)~{:}{-}~\textit{person}(X),~\text{not}~
  \textit{allergic}(X). 
\end{equation*}

The following notation in Prolog (Programming in logic) mimics the logical implication:

\begin{equation*}
  \forall X: \textit{person}(X) \wedge \overline{\textit{allergic}(X)}
  \implies \textit{likes\_sushi}(X)
\end{equation*}

In this thesis we use modern Logic Programming engines, such
as \acrlong{asp} \parencite{ASPbook,whatisasp} and \acrlong{idp}
\parencite{idp} %(we explain them in details in the following chapter)
that naturally support constraints and various types of logical
inference.


\section{Contribution}
Given the abundance of relational data (such as spreadsheets, graphs
or logical structures) and of various machine learning and data
problems, a natural question arises: \textit{``How can we model
relational data mining problems?''}. This thesis introduces a number
of foundational relational problems and offers a uniform view on them.
Since the data and problems are relational, we approach them using
relational methods, namely, logic programming. [is it the only relational method?? ``namely" implies it is, but i'm not sure it is true...]

The \textbf{main contribution} of this thesis is in the novel problems
it offers. [the sentence is out of context here, move it]

Concretely, the thesis investigates the following questions:
\begin{itemize}
  \item \cone: What are the challenges and advantages of generalizing
    classic data mining problems, such as the Boolean Matrix
    Factorization problem, into the relation setting?
  \item \ctwo: What practical problems are intrinsically relational and
    how can they be modelled and used in practice?
  \item \cthree: \textit{(the inverse problem)} given a relational
    model, can we mark some parts of its to be uncertain and in what
    setting would it be applicable?
  \item \cfour: 
%  Similarly to unstructured constraint-based itemset mining: 
    can we apply declarative relational mining
    techniques to structured patter mining, such as sequence, graph
    and query mining?
\end{itemize}

Addressing \cone, we look into the classical problem of matrix
factorization. We demonstrate how the problem of Boolean Matrix
Factorization can be generalized into relational setting, called
\textit{Relational Data Factorization}, and
demonstrate how it allows to model a variety of classic data mining
problems and novel tasks as well.

With regards to \ctwo, we look into a more practical setting ;
Typically, machine learning and data mining algorithms make the i.i.d.
(independent identically distributed) assumption, which of course
breaks in the relational setting. If we look at a spreadsheet, it is
typically one or more tables with relationships between them, usually
in a form of a formula, a macro or of some other constraint. If we
want to detect these connections between tables or their parts, we
have to give up the i.i.d. assumption. This setting, referred to as the
\textit{Tabular constraint learning}, breaks the standard
machine learning and data mining algorithms and 
makes the problem essentially relational.

This leads us to \cthree: if we already have a relational model but
something changes over time or due to changes in the specification, then we would like
to be able to indicate uncertainty in some parts of the model. These
uncertain or, \textit{open}, parts are called \textit{sketched} and
the whole sketched model is referred to as a \textit{sketch} and the related
problem as \textit{constraint sketching}.

Finally, lwith respect to \cfour, if we look at the area of itemset mining, where \acrlong{cp}
has been successfully applied, then the natural question to ask is:
if \acrshort{cp} has been successful for unstructured, or
non-relational, pattern mining, can we model its \textit{structured},
or \textit{relational}, version using logic programming? To answer
this question we look into \textit{sequence}, \textit{graph} and
\textit{query mining} and attack them using logic programming and
hybrid approaches based on logic programming.


\section{Structure of the Thesis}
\textbf{Chapter} \ref{ch:background} introduces background information on Logic, Logic and Answer Set Programming and Pattern Mining.

\textbf{Chapter} \ref{ch:ReDF} presents the novel problem of Relational Data
Factorization. It demonstrates how this problem generalizes various
data mining problems such as Database Tiling and Boolean Matrix
Factorization. We start with the most basic setting and demonstrate
how various problems can be modelled within the framework by adding
or modifying the constraints. Then, we show how the framework can be
used for new and interesting applications and propose Answer Set
Programming as a method to solve the problem in the general case.
For each problem we provide an extensive experimental evidence,
including solver parameter tuning. The chapter consists of the
research previously published in the following paper:

\begin{addmargin}[2em]{2em}

Sergey Paramonov,  Matthijs van Leeuwen, Luc De Raedt: Relational data
factorization, Machine Learning, Springer, 2017.

\end{addmargin}



\textbf{Chapter} \ref{ch:TaCLe} introduces  the novel problem of
Tabular Constraint Learning and the system, called TaCLe (pronounced
as the word ``tackle''). The problem setting is straightforward to
explain: we have an Excel file with formulae. The file is imported as
CSV. The formulae are lost. Can we reconstruct them? We can already 
see that the problem setting is different from the standard machine
learning problem. In a spreadsheet, columns are no longer variables
and rows are no longer records. Textual and numeric data are mixed.
Spreadsheet functions, like Fuzzy-lookup, are unorthodox and unseen in
the constraint programming and learning research communities.

The chapter consists of the research previously published in the following papers:

\begin{addmargin}[2em]{2em}
Samuel Kolb (*), Sergey Paramonov (*), Tias Guns, Luc De Raedt:
  Learning constraints in spreadsheets and tabular data. Machine
  Learning 106(9-10), 2017.


Sergey Paramonov (*), Samuel Kolb (*), Tias Guns, Luc De Raedt:
TaCLe: Learning constraints in tabular data. CIKM. Sheridan
Communications, 2017.
\end{addmargin}


\textbf{Chapter} \ref{ch:Sketching} introduces the novel problem of
Sketched Answer Set Programming. The idea of sketching takes root in
imperative programming, when a function in C or Java has a part of it,
typically a constant left unspecified. A user then provides a number
of examples for this sketched constant to be filled in with a
correct value satisfying the examples. Inspired by this approach, we
introduce Sketched Answer Set Programming, where a user writes a
regular program but with certain parts such as constants, predicates
or operators are left uncertain. Then, we rewrite this sketched
program into a regular ASP program, i.e., we use a standard ASP solver
to complete sketches.


The chapter consists of the research previously published in the following papers:
\begin{addmargin}[2em]{2em}
  Sergey Paramonov, Christian Bessiere, Anton Dries, Luc De Raedt:
  Sketched Answer Set Programming. CoRR abs/1705.07429, 2017.
\end{addmargin}


\textbf{Chapter} \ref{ch:StructuredMining} introduces a logic
based approach to structured pattern mining. The idea to apply general
solvers to pattern mining takes root in the work of
\cite{declrativeapproach}, where Constraint Programming is applied to
Itemset Mining. Contrary to itemset mining, where objects do not have
any particular structures, structured mining studies mining patterns
in sequences, trees, graphs, etc. First, we introduce a general
mathematical model of graph mining based on the logical formulation of
the constraints. Second, we show that the model can be hybridized by
using our generic framework together with specialized solvers
developed for particular pattern mining problems.

The chapter consists of the research previously published in the following papers:
\begin{addmargin}[2em]{2em}
Sergey Paramonov, Matthijs van Leeuwen, Marc Denecker, Luc De Raedt:
An Exercise in Declarative Modeling for Relational Query Mining. ILP, 
2015.


Sergey Paramonov, Daria Stepanova, Pauli Miettinen:
Hybrid ASP-Based Approach to Pattern Mining. RuleML+RR, 2017.
\end{addmargin}

\textbf{Chapter} \ref{ch:conclusions} summarizes the thesis and
discusses future work possibilities.

\section{Datasets, Code and Experimental Results}
In order to make results repeatable and to follow the Open Source
spirit, we have opened and published the datasets, use-cases,
meta-data and other useful information.
They can be found in the following GitHub repositories:
\begin{itemize}
\item \url{https://github.com/SergeyParamonov/TaCLe}
\item \url{https://github.com/SergeyParamonov/sketching}
\item \url{https://github.com/SergeyParamonov/LGM}
\end{itemize}
\cleardoublepage
% vim: tw=70 nocindent expandtab foldmethod=marker foldmarker={{{}{,}{}}}

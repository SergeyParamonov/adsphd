\chapter{Introduction}\label{ch:introduction}
\epigraph{
- Homer, is it the way you pictured PhD life?\\
- Yeah, pretty much, except we drove in a van solving mysteries.
}{Could be Homer Simpson}

\begin{addmargin}[2em]{2em}
This thesis links the research fields of data mining, machine learning
and logic
programming. We introduce them in turn and provide an overview of the
contributions of the thesis and
its general structure.
\end{addmargin}

\section{Data Mining and Machine Learning}

\section{Logic Programming}
In his seminal work Sir Bob \textcite{kowalski} proclaimed:
\begin{center}
  Algorithm = Logic + Control.
\end{center}

\section{Contribution}
Given the abundance of relational data (such as spreadsheets, graphs
or logical structures) and of various machine learning and data
problems. A natural question arises: \textit{``How can we model
relational data mining problems?''}
The contribution of this thesis is manifold.

\begin{itemize}
  \item \cone: What are the challenges and advantages of generalizing
    classic data mining problems, such as the Boolean Matrix
    Factorization problem, into the relation setting?
  \item \ctwo: What practical problems are intrinsically relational and
    how can they be modelled and used in practice?
  \item \cthree: Looking at the inversed problem: given a relational
    model, can we mark some parts of its to be uncertain and in what
    setting would it be applicable?
  \item \cfour: Similarly to unstructured constraint-based itemset
    mining: can we apply declarative relational mining
    techniques to structured patter mining, such as sequence, graph
    and query mining?
\end{itemize}



\section{Structure of the Thesis}
\textbf{Chapter} \ref{ch:background} introduces background information on Logic, Logic and Answer Set Programming and Pattern Mining.

\textbf{Chapter} \ref{ch:ReDF} presents the novel problem of Relational Data
Factorization. It demonstrates how this problem generalizes various
data mining problems such as Database Tiling and Boolean Matrix
Factorization. We start with the most basic setting and demonstrate
how various problems can be modelled within the framework by adding
or modifying the constraints. Then, we show how the framework can be
used for new and interesting applications and propose Answer Set
Programming as a method to solve the problem in the general case.
For each problem we provide an extensive experimental evidence,
including solver parameter tuning. The chapter consists of the
research previously published in the following paper:

\begin{addmargin}[2em]{2em}

Sergey Paramonov,  Matthijs van Leeuwen, Luc De Raedt: Relational data
factorization, Machine Learning, Springer, 2017

\end{addmargin}



\textbf{Chapter} \ref{ch:TaCLe} introduces  the novel problem of
Tabular Constraint Learning and the system, called TaCLe (pronounced
as the word ``tackle''). The problem setting is straightforward to
explain: we have an Excel file with formulae. The file is imported as
CSV. The formulae are lost. Can we reconstruct them? We can already 
see that the problem setting is different from the standard machine
learning problem. In a spreadsheet, columns are no longer variables
and rows are no longer records. Textual and numeric data are mixed.
Spreadsheet functions, like Fuzzy-lookup, are unorthodox and unseen in
the constraint programming and learning research communities.

The chapter consists of the research previously published in the following papers:

\begin{addmargin}[2em]{2em}
Samuel Kolb (*), Sergey Paramonov (*), Tias Guns, Luc De Raedt:
  Learning constraints in spreadsheets and tabular data. Machine
  Learning 106(9-10): 1441-1468 (2017)


Sergey Paramonov (*), Samuel Kolb (*), Tias Guns, Luc De Raedt:
TaCLe: Learning constraints in tabular data. CIKM. Sheridan
Communications, 2017.
\end{addmargin}


\textbf{Chapter} \ref{ch:Sketching} introduces the novel problem of
Sketched Answer Set Programming. The idea of sketching takes root in
imperative programming, when a function in C or Java has a part of it,
typically a constant left unspecified. A user then provides a number
of examples for this sketched constant to be filled in with a
correct value satisfying the examples. Inspired by this approach, we
introduce Sketched Answer Set Programming, where a user writes a
regular program but with certain parts such as constants, predicates
or operators are left uncertain. Then, we rewrite this sketched
program into a regular ASP program, i.e., we use a standard ASP solver
to complete sketches.


The chapter consists of the research previously published in the following papers:
\begin{addmargin}[2em]{2em}
  Sergey Paramonov, Christian Bessiere, Anton Dries, Luc De Raedt:
  Sketched Answer Set Programming. CoRR abs/1705.07429 (2017)
\end{addmargin}


\textbf{Chapter} \ref{ch:StructuredMining} introduces a logic
based approach to structured pattern mining. The idea to apply general
solvers to pattern mining takes root in the work of
\cite{declrativeapproach}, where Constraint Programming is applied to
Itemset Mining. Contrary to itemset mining, where objects do not have
any particular structures, structured mining studies mining patterns
in sequences, trees, graphs, etc. First, we introduce a general
mathematical model of graph mining based on the logical formulation of
the constraints. Second, we show that the model can be hybridized by
using our generic framework together with specialized solvers
developed for particular pattern mining problems.

The chapter consists of the research previously published in the following papers:
\begin{addmargin}[2em]{2em}
Sergey Paramonov, Matthijs van Leeuwen, Marc Denecker, Luc De Raedt:
An Exercise in Declarative Modeling for Relational Query Mining. ILP
2015: 166-182


Sergey Paramonov, Daria Stepanova, Pauli Miettinen:
Hybrid ASP-Based Approach to Pattern Mining. RuleML+RR 2017: 199-214
\end{addmargin}

\textbf{Chapter} \ref{ch:conclusions} summarizes the thesis and
discusses future work possibilities.

\section{Datasets, code and experimental results}
In order to make results repeatable and to follow the Open Source
spirit, we have opened and published the datasets, use-cases,
meta-data and other useful information.

They can be found in the following GitHub repositories:
\begin{itemize}
\item \url{https://github.com/SergeyParamonov/TaCLe}
\item \url{https://github.com/SergeyParamonov/sketching}
\item \url{https://github.com/SergeyParamonov/LGM}
\end{itemize}
\cleardoublepage
% vim: tw=70 nocindent expandtab foldmethod=marker foldmarker={{{}{,}{}}}

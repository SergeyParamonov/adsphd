\chapter*{Beknopte samenvatting}
\textbf{Het Modelleren van Relationele Datamining}


Het doel van datamining is om nieuwe kennis te ontdekken in data.
Er werden vele modellen, systemen en technieken voor ontwikkeld.
Opdat deze effici\"ent zouden werken, moet aan een aantal veronderstellingen voldaan zijn.  Een gebruikelijke veronderstelling is dat de records in de data onafhankelijk zijn van elkaar.
In complexe data zoals rekenbladen en relationele gegevensbanken klopt deze veronderstelling echter niet. Het modelleren van relationele-datamining is dus niet triviaal. Dit doctoraatsproefschrift bestudeert een aantal problemen in relationele-datamining en toont hoe ze kunnen worden gemodelleerd en opgelost.

In het bijzonder volgen we een algemene, declaratieve kijk op een klasse van problemen in relationele-datamining. Een declaratieve benadering houdt in dat men het probleem zelf beschrijft, in plaats van hoe het op te lossen. 

De eerste bijdrage in dit proefschrift is dat het aantoont hoe het klassiek probleem van de factorisatie van Booleaanse matrices kan worden veralgemeend naar de factorisatie van relationele data, en toont dat hiermee een breed spectrum aan problemen kan worden gemodelleerd.
De tweede bijdrage in dit proefschrift zit in het nieuwe probleem van het leren van constraints in tabellen, waarbij een algoritme de formules moet reconstrueren in rekenbladen waarvan men enkel de getallen nog heeft.
De derde bijdrage is de introductie van `programmeren met geschetste
antwoordverzamelingen' (sketched answer set programming). Hierbij kan men delen van een logisch programma markeren als onzeker, of open, en het volledige programma laten reconstrueren met behulp van voldoende positieve en negatieve voorbeelden.  Tenslotte toont het hoe onze aanpak kan worden toegepast op allerlei gestructureerde en ongestructureerde patroonherkenningsproblemen.

Samenvattend: we onderzoeken hoe een aantal belangrijke problemen in
relationele datamining kunnen worden gemodelleerd met Answer Set
Programming en FO(.).


%%%%%%%%%%%%%%%%%%%%%%%%%%%%%%%%%%%%%%%%%%%%%%%%%%
% Keep the following \cleardoublepage at the end of this file, 
% otherwise \includeonly includes empty pages.
\cleardoublepage

% vim: tw=70 nocindent expandtab foldmethod=marker foldmarker={{{}{,}{}}}

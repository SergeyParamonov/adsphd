\chapter{Acknowledgements} \label{ch:ack} I would like to say a big THANK YOU to everyone who has been involved in making this happen. To all my friends and colleagues who have been an enormous support to me.  

First and foremost, I would like to thank Luc De Raedt, my promoter. My scientific perception is shaped by his vision and expertise that he shared with me over the years. This work would not have been possible without his help and ideas. Working with him has taught me to see the bigger picture, and find aspirations worth spending my time and energy on. From him, I learned what it is like to be a researcher.

A special thank you goes to Marc Denecker. It might seem that Marc is always in the office, but actually, he is usually in a beautiful mathematical world that we do not get to see. This work would not have been possible without his many presents -- ideas from that world -- which he has given me. I am extremely grateful to him for sharing this with me.

I am also grateful to the members of my jury. I am glad that I worked with Matthijs van Leeuwen at the beginning of my PhD. He introduced me the field of data mining, and his comments and advice on the design and presentation of experiments have been priceless. Many insights in modeling have come from working with Gerda Janssens; I am grateful for all the discussions we had on IDP and graph mining. The work on sketching would not have been possible without great help and advice on constraint modelling from Christian Bessiere. Thanks as well to Helmut Simonis: my knowledge of constraint learning and my thesis have improved a lot. I am also grateful to Karl Meerbergen for his insights and comments on relational factorization. And last but not least, I would like to thank Hilde Heynen for being my chairman.

I have had the great pleasure of working with wonderful people at the Computer Science Department. They were not just colleagues, but friends as well. Irma and Anton were the main forces driving our small community of friends at the department. Jessa is the person binding it all together in Leuven. I am really honoured to be her friend. Francesco -- kernels. Thanks Evgeniya for all the wonderful weekend coffees we had. Sebastijan and Nikolina for the summer school and all the fun events they have organized for us. I am grateful to Kurt for his help that always came when needed. To Samuel, for all the great work we have done together, and for making it fun to be in the office. To Ondrej and Gitte, for showing that you can always come back after you have disappeared. To Pedro, for telling us all the stories that happen to him. To Antoine, for his enthusiastic support of all our events in Leuven. To Anna, for showing us geocaching. To Behrouz for teaching us about introverts. To Vladimir, for being the first person to welcome me in Leuven and show me around, and to Leen for taking care of him. To Stefano for showing me the most random object in the world. I still completely accidentally remember it from time to time. To Joana, for suddenly appearing and saving Anton. To Mohit, Nitesh, Arrchit, Guy, Tim, Vincent, Tom, Jonas, Killian, Ellia, Robin, Bogdan for all the coffees we had at the department. A special thank you is to Tias, for being simultaneously colleague, advisor, friend, FOSDEM org and circus teacher. To Ben, for showing that you can complain all the time, and \textit{still} look like a happy person while doing it. To Siegfried, for guiding me through the pattern mining world while playing volleyball and telling all his conference travelling stories.

Ever since I started joining volleyball on Thursday evenings, it has had a special place in my heart. Not only because of the game, but because of all the wonderful people I met there: Luc, Wim, Bertram, Inge, Hanne, Marteen, Vera, Frank, Bram, Irena, Cyrielle, Arturo, and everyone else. I plan to continue this tradition. (But not every Thursday -- Antwerp is far!)

I am really grateful to my EMCL friends who have been my support for many years already. To Alina for all her help, and our shared journeys, and stories. It is always great to have friends like her. To Peter and Ronald: we might not see each other for years, but once we meet again, it feels like we only saw each other yesterday. 

Thanks to the people at Sentiance who showed great interest in the research I have done. To Vincent and Bertrand, who believed in me even before I finished this thesis. And to Jean, for being a great help, especially with the random things.

To my Phystech friends, who keep in touch after all these years. Maxim, Michael, Konstantin, Nick, Ilya, Alexandra, Nastya and Yuri: every time I travel to Russia, I am always happy to see you.

Throughout my PhD, I have met a lot of interesting people, especially at various events around the world. I am grateful to Tamara from the Summer School of Science for showing me that chemistry can be fun, and supporting me through my thesis writing frustrations. To Raffael and Agnieszka for showing me London during my conference stay. To Aleksandra for talking and being there for me when I really needed it. To Daria and Pauli for believing in my ideas in structured pattern mining. To Anastasia for all the workshops and hackatons she has organized. To Nastya, for our walks in Saint-Petersburg, and organizing a wonderful wedding party. To Wouter and Seraphine for making me feel at home in Leuven.

When I was six years old, I met Kirill and Olga. You have always been a great support in everything I do, and you still are.

To Alexander Isaakovich, my first real math teacher, for introducing me to the world of mathematics.

Last, but not least, I would like to thank my family. Irina made me the person I am today. She is my constant support, and I could not have done this without her. Thanks to Olga for supporting and talking to me. To Anja and Andrei, for being wonderful visitors and great hosts. And my infinite gratitude goes to my parents for creating an environment in which I could grow and realize what I wanted in life.


\chapter{Abstract} \label{ch:abstract}
\comment{Overall message?}

\comment{What data mining is}\\
Data Mining is the process of discovering new knowledge in data.
The Data Mining research community makes use of the  \textit{data
independence assumption}, i.e., which states that the collected data points are independent from each other.

\comment{What relational means}\\
However, in reality, it is often the case that the objects are 
intrinsically connected and related to each other by means of relations. 
This setting is called \textit{relational} and the associated task 
is referred to as \textit{relational learning}.

\comment{Relational problems}\\
Electronic tables, spreadsheets, and databases are all examples of
relational data that are in wide use today. The objects are connected
by means of relations and constraints. These are
tables and formulae in spreadsheets, and  schema relations and integrity
constraints in databases.

\comment{Key issue}\\
We argue that a general approach and characterization for modelling and solving data mining  
problems in the relational setting are missing. The goal of this thesis
is to fill in this gap.


\comment{Contribution}\\
First, we demonstrate how constraint learning problem in the relation
setting is different from the classical machine learning, and we
propose a system named TaCLe as the first working in this setting.

Secondly, we demonstrate how relational approach can be used to
generalize the classical problem of Boolean Matrix Factorization into the Relational
Data Factorization. This allows one to model a spectrum of classical data
mining problems and introduces new ones as well.

Thirdly, we demonstrate how existing relational reasoning formalisms,
such as Answer Set Programming, can be enhanced by sketching, a relational learning
technique.

Finally, we demonstrate how relational approaches can be used
to mine relational patterns. This is known as structured pattern mining in
data mining community.


%%%%%%%%%%%%%%%%%%%%%%%%%%%%%%%%%%%%%%%%%%%%%%%%%%
% Keep the following \cleardoublepage at the end of this file, 
% otherwise \includeonly includes empty pages.
\cleardoublepage

% vim: tw=70 nocindent expandtab foldmethod=marker foldmarker={{{}{,}{}}}

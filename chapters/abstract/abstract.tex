\chapter{Abstract} \label{ch:abstract}
The goal of data mining is to discover new knowledge in the data. Many models, systems and techniques have been developed. For these systems and models to work efficiently a number of assumptions must hold: a common assumption is that of  data independence. However, in complex and interconnected data, such as spreadsheets or relational tables in databases, this assumption often does not hold. This makes modelling relational data mining problems non-trivial. This thesis studies a number of relational data mining problems and demonstrates how they can be modelled and solved.

Specifically, we follow a general, \textit{declarative}, view on a
class of relational data mining problems using the principles of logic
programming. In a declarative approach one specifies \textit{what} the
problem is, instead of \textit{how} to solve it. Logic programming offers a methodology in which a problem is formulated using logical rules and an inference engine finds an answer. 

The first contribution of this thesis is that it shows how a classic
problem of Boolean Matrix Factorization can be generalized to the
relational setting of \textit{Relational Data Factorization}, and 
demonstrates how it can model a wide range of data mining problems. The
second contribution of this thesis is in the novel
problem of \textit{Tabular Constraint Learning}, where one recovers
Excel-like formulae in spreadsheets. The third contribution of this
thesis is the introduction of \textit{Sketched Answer Set
Programming}, which allows one to mark parts of a logic program as
\textit{uncertain}, or \textit{open}, and recover, or
\textit{synthesize}, the program using a number of positive and
negative examples. Finally, it demonstrates how our approach can be
applied to model various structured and unstructured pattern mining
problems.

\pubrev
To summarize, we investigate how a number of important relational data
mining problems can be modelled using inductive logic programming,
based on modern logic programming such as Answer Set Programming and FO(.). 
\pubrevend

%%%%%%%%%%%%%%%%%%%%%%%%%%%%%%%%%%%%%%%%%%%%%%%%%%
% Keep the following \cleardoublepage at the end of this file, 
% otherwise \includeonly includes empty pages.
\cleardoublepage

% vim: tw=70 nocindent expandtab foldmethod=marker foldmarker={{{}{,}{}}}

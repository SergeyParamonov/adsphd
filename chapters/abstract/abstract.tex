\chapter{Abstract} \label{ch:abstract}
This goal of Data Mining is to discover new knowledge in the data. Many models, systems and techniques have been developed. For these systems and models to work efficiently a number of assumptions must hold: a common assumption is that of the data independence. However, in complex and interconnected data, such as spreadsheets or relational tables in databases, this assumption often does not hold. This makes modelling relational data mining problems non-trivial. This thesis studies a number of relational data mining problems and demonstrates how they can be modelled and solved.

In this thesis we propose a general, \textit{declarative}, view on a class of relational data mining problems using Logic Programming. In a declarative approach one specifies \textit{what} is the problem, instead of specifying \textit{how} to solve it. Logic Programming offers a methodology in which a problem is formulated using logical rules and an inference engine finds an answer. 

The first contribution of this thesis is that it shows how a classic problem of Boolean Matrix Factorization can be generalized into the relational setting and model a class of data mining problems, we have named this problem \textit{Relational Data Factorization}. The second contribution of this thesis is that we propose the novel problem of \textit{Tabular Constraint Learning}, where one recovers Excel-like formulae in spreadsheets, and an Inductive Logic Programming-based approach to find them in raw textual and numeric data. The third contribution of this thesis is in the introduction of \textit{Sketched Answer Set Programming}, which allows one to mark parts of a logic program as \textit{uncertain}, or \textit{open}, and recover, or \textit{synthesize}, the program using a number of positive and negative examples. Finally, we apply our modeling approach to pattern mining and show how it can be used to model a number of structured, such as sequence, graph and query, pattern mining problems as well as unstructured, i.e., itemset mining. 

Therefore, we investigate how a number of important relational data mining problems can be modelled using a unified relational view, based on logic programming. We conclude on this with open problems and remaining computational challenges.

%%%%%%%%%%%%%%%%%%%%%%%%%%%%%%%%%%%%%%%%%%%%%%%%%%
% Keep the following \cleardoublepage at the end of this file, 
% otherwise \includeonly includes empty pages.
\cleardoublepage

% vim: tw=70 nocindent expandtab foldmethod=marker foldmarker={{{}{,}{}}}

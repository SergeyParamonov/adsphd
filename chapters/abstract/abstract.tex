\chapter{Abstract} \label{ch:abstract}
The goal of data mining is to discover new knowledge in the data. This
thesis studies a number of relational data mining problems and
demonstrates how they can be modelled and solved. Relatonal data
mining involves dealing with complex and interconnected data, such as
spreadsheets or relational tables in databases. 

Specifically, we follow a general, \textit{declarative}, view on a
class of relational data mining problems using the principles of logic
programming. In a declarative approach one specifies \textit{what} the
problem is, instead of \textit{how} to solve it. 

The contributions of this thesis are 1) it shows how Boolean Matrix Factorization can be generalized to the
relational setting of \textit{Relational Data Factorization}, and 
demonstrates how Relational Data Factorization can model a wide range
of data mining problems; 2) it introduces the novel
problem of \textit{Tabular Constraint Learning}, where one recovers
Excel-like formulae in spreadsheets; 3) it introduces the problem of \textit{Sketched Answer Set
Programming}, which allows one to mark parts of a logic program as
\textit{uncertain}, or \textit{open}, and recover, or
\textit{synthesize}, the program using a number of positive and
negative examples; 4) it demonstrates how our approach can be
applied to model various structured and unstructured pattern mining
problems under constraints.

\pubrev
To summarize, we investigate how a number of important relational data
mining problems can be modelled using Answer Set Programming and FO(.). 
\pubrevend

%%%%%%%%%%%%%%%%%%%%%%%%%%%%%%%%%%%%%%%%%%%%%%%%%%
% Keep the following \cleardoublepage at the end of this file, 
% otherwise \includeonly includes empty pages.
\cleardoublepage

% vim: tw=70 nocindent expandtab foldmethod=marker foldmarker={{{}{,}{}}}

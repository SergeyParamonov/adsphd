\chapter{Abstract} \label{ch:abstract}
\comment{What data mining is}\\
Data Mining is the process of discovering new knowledge from the data.
A significant attention in the research community is devoted to data
analysis in the presence of the data independence assumption, i.e.,
when the collected data points are independent from each other.

\comment{What relational means}\\
However, it is often the case that the objects are connected and
related to each other by means of relations. This setting is called
\textit{relational} and the associated task is referred as
\textit{relational reasoning}.

\comment{Relational problems}\\
Electronic tables, spreadsheets, and databases are all examples of
relational data that is in a wide use today. The objects are connected
by means of relations and constraints. In spreadsheets these are
tables and formulae, and in databases schema relations and integrity
constraints.

\comment{Key issue}\\
We argue that a general approach for modeling and solving data mining  
problems in the relational setting is missing. The goal of this thesis
is to fill in this gap.


\comment{Contribution}\\
Firstly, we demonstrate how the problem of learning in relation
setting is different from the classical machine learning approach and
propose a system named TaCLe as the first working in this setting.

Secondly, we demonstrate how the relational approach generalizes a
classical problem of boolean matrix factorization into the Relational
Data Factorization, which allows to model a spectrum of classical data
mining problems and introduces new ones as well.

Thirdly, we demonstrate how existing relational reasoning formalisms,
such as Answer Set Programming, can be enhanced by relational learning
techniques known as sketching.

Last but not least, we demonstrate how relational approach can be used
to mine relational patterns, known as structured pattern mining in
data mining community.


%%%%%%%%%%%%%%%%%%%%%%%%%%%%%%%%%%%%%%%%%%%%%%%%%%
% Keep the following \cleardoublepage at the end of this file, 
% otherwise \includeonly includes empty pages.
\cleardoublepage

% vim: tw=70 nocindent expandtab foldmethod=marker foldmarker={{{}{,}{}}}

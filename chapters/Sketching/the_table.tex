\newcommand{\rulesep}{\unskip\ \vrule\ }
\newcommand{\hrulesep}{\unskip\ \hrule\ }

\begin{figure*}[htb]
%  \captionsetup{font=captionfontsize}
\begin{subfigure}[t]{0.348\textwidth}
  \renewcommand{\figurename}{Listing}
  \vspace{3pt}
  \begin{Verbatim}[fontsize=\scriptsize,numbers=left,xleftmargin=-2mm,commandchars=\\\{\}]
[SKETCH]
reached(Y) :- cycle(a,Y).\label{line:reached}
reached(Y) :- cycle(X,Y),\label{line:reached2}
                     reached(X).
        :- ?p(Y), ?not ?q(Y). \label{line:ham_constraint}
[EXAMPLES]
positive:cycle(a,b).cycle(b,c)...
negative:cycle(a,b).cycle(b,a).
[SKETCHEDVAR]
?p/1 : node, reached
?q/1 : node, reached
[FACTS]
node(a). node(b). node(c).
[EXAMPLES]
positive: cycle(a,b).cycle(b,c).\label{line:example}
                     cycle(c,a).
negative: cycle(a,b).cycle(b,a).
\end{Verbatim}
\caption{Hamiltonian Cycle\\ASP encoding from\\ \protect\parencite{ASPbook}} \label{lst:ham}
\end{subfigure}
\rulesep
\begin{subfigure}[t]{0.645\textwidth}
  \vspace{3pt}
  \renewcommand{\figurename}{Sketch}
  \begin{Verbatim}[fontsize=\scriptsize,commandchars=\\\{\}]
%%%%% EXAMPLES AND DECISIONS %%%%%%
positive(0). cycle(0,a,b). cycle(0,b,c). cycle(0,c,a). \label{line:example_rewritten}
reified_q_choice(c_node). reified_q_choice(c_reached).
1 \{decision_q(X) :  reified_q_choice(X)\} 1. \label{line:decision1}
1 \{decision_not(positive); decision_not(negative)\} 1.\label{line:decision2}
%%%%% INFERENCE RULES %%%%%%
reified_q(E,c_node,X0) :- node(X0),examples(E). \label{line:reified_q}
reified_q(E,c_reached,X0) :- reached(E,X0).
reached(E,Y) :- cycle(E,a,Y),examples(E). \label{line:reach_rewritten}
reached(E,Y) :- cycle(E,X,Y),reached(E,X),examples(E).
reified_not(E,positive,Q,Y) :- reified_q(E,Q,Y). \label{line:reified_not1}
reified_not(E,negative,Q,Y) :- not reified_q(E,Q,Y), \label{line:reified_not2}
                                            dom1(Y),...
%%%%% POSITIVE/NEGATIVE SKETCHED RULES %%%%%%
:- reified_p(E,P,Y),reified_not(E,Not_D,Q,Y),\label{line:positive_rewritten1}
decision_p(P),decision_q(Q),decision_not(Not_D),positive(E).\label{line:positive_rewritten2}
neg_sat(E) :- reified_p(E,P,Y),reified_not(E,Not_D,Q,Y), \label{line:negative_rewritten1}
decision_p(P),decision_q(Q),decision_not(Not_D),negative(E). \label{line:negative_rewritten2}
:- not neg_sat(E), negative(E).
\end{Verbatim}
\caption{ASP core of rewritten Listing \protect\ref{lst:ham}\\(only predicate $q$ shown, domains, facts, etc omitted)} \label{lst:rewriting}
\end{subfigure}
\hrulesep
\begin{subfigure}[t]{0.49\textwidth}
  \renewcommand{\figurename}{Sketch}
  \vspace{2pt}
\begin{Verbatim}[fontsize=\scriptsize,numbers=left,xleftmargin=0mm]
[SKETCH] %constraints on squares,rows..
:- cell(X,Y,N), cell(X,Z,M), Y ?= Z, N ?= M.
:- cell(X,Y,N), cell(Z,Y,M), X ?= Z, N ?= M.
insquare(S,N) :- cell(X,Y,N), square(S,X,Y).
:- num(N), squares(S), ?not insquare(S,N). 
\end{Verbatim}
\caption{Sudoku (core). ASP code from\\\protect \parencite{asp_tutorial_sudoku}} \label{lst:sudoku}
\end{subfigure} 
\rulesep
%  \begin{subfigure}[t]{0.32\textwidth}
%   \vspace{2pt}
%     \renewcommand{\figurename}{Listing}
% \begin{Verbatim}[fontsize=\scriptsize]
% [SKETCH]
% :- q(w,Rw,Cw), q(b,Rb,Cb), Rw ?= Rb.
% :- q(w,Rw,Cw), q(b,Rb,Cb), Cw ?= Cb.
% :- q(w,Rw,Cw), q(b,Rb,Cb), 
%             Rw ?+ Rb ?= Cw ?+ Cb.
% [EXAMPLES]
% positive: q(w,1,1). q(w,2,2). ...
% negative: q(b,2,2). q(w,3,1). ...
% \end{Verbatim}
% \caption{B\&W Queens sketch ($q$ for \textit{queen}; core)}\label{lst:queens}
% \end{subfigure} 
% \rulesep
\begin{subfigure}[t]{0.44\textwidth}
  \vspace{2pt}
  \renewcommand{\figurename}{Listing}
\begin{Verbatim}[fontsize=\scriptsize]
[SKETCH] % constraints on rows and columns
:- cell(X,Y,N), cell(X,Z,M), Y ?= Z, N ?= M.
:- cell(X,Y,N), cell(Z,Y,N), X ?= X, N ?= M.
[EXAMPLES]
positive: cell(1,1,a). cell(1,2,b). 
                             cell(1,3,c)...
\end{Verbatim}
\caption{Latin Square \\(based on Sudoku; core)} \label{lst:latin_square}
\end{subfigure}

\caption{Collection of sketches and an example of rewriting used in the paper}
\label{fig:table_with_sketches}
\end{figure*}
